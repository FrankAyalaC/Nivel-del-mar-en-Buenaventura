%% The '5p' and 'times' class options of elsarticle are used for Elsevier CRC
\documentclass[5p,times,authoryear]{elsarticle}

%% The `ecrc' package must be called to make the CRC functionality available
%% ecrc_RIAI es el paquete ecrc de Elsevier con modificaciones para la revista RIAI
\usepackage{ecrc_RIAI}
\usepackage[dvipsnames]{xcolor}
\usepackage{parskip}

%% The ecrc package defines commands needed for running heads and logos.
%% For running heads, you can set the journal name, the volume, the starting page and the authors
%%%%%%%%%%%%%%%%%%%%%%%%%%%%%%%%% Aaadido por Secretaraa RIAI

\usepackage[spanish]{babel}

%\usepackage[spanish]{babel}
\usepackage[utf8]{inputenc}% Idioma

\usepackage{amsmath}            % Para las referencias a ecuaciones con \eqref
\usepackage{epstopdf}           % Para poder insertar figuras .eps al compilar con PDFLATEX
\usepackage{flushend}           % Para igualar las columnas de la altima pagina
%\usepackage{hyperref}           % Para hipervanculos dentro del PDF
%%%%%%%%%%%%%%%%%%%%%%%%%%%%%%%%%%%%%%%%%%%%%%%%%%%%%%%

% set the starting page if not 1
\firstpage{1}

%% Give the author list to appear in the running head
%% Example \runauth{C.V. Radhakrishnan et al.}
\runauth{Ayala Franklin et al.}

%% The choice of journal logo is determined by the \jid and \jnltitlelogo commands.
%% A user-supplied logo with the name <\jid>logo.pdf will be inserted if present.
%% e.g. if \jid{yspmi} the system will look for a file yspmilogo.pdf
%% Otherwise the content of \jnltitlelogo will be set between horizontal lines as a default logo

%% Give the abbreviation of the Journal. Contast the Publisher if in doubt what this is.
%\jid{RIAI}

%% Give a short journal name for the dummy logo (if needed)
%\jnltitlelogo{}

%% Hereafter the template follows `elsarticle'.
%% For more details see the existing template files elsarticle-template-harv.tex and elsarticle-template-num.tex.

%% Elsevier CRC generally uses a numbered reference style
%% For this, the conventions of elsarticle-template-num.tex should be followed (included below)
%% If using BibTeX, use the style file elsarticle-num.bst

%% End of ecrc-specific commands
%%%%%%%%%%%%%%%%%%%%%%%%%%%%%%%%%%%%%%%%%%%%%%%%%%%%%%%%%%%%%%%%%%%%%%%%%%

%% The amssymb package provides various useful mathematical symbols
\usepackage{amssymb}
%% The amsthm package provides extended theorem environments
%% \usepackage{amsthm}

%% The lineno packages adds line numbers. Start line numbering with
%% \begin{linenumbers}, end it with \end{linenumbers}. Or switch it on
%% for the whole article with \linenumbers after \end{frontmatter}.
%% \usepackage{lineno}

%% natbib.sty is loaded by default. However, natbib options can be
%% provided with \biboptions{...} command. Following options are
%% valid:

%%   round  -  round parentheses are used (default)
%%   square -  square brackets are used   [option]
%%   curly  -  curly braces are used      {option}
%%   angle  -  angle brackets are used    <option>
%%   semicolon  -  multiple citations separated by semi-colon
%%   colon  - same as semicolon, an earlier confusion
%%   comma  -  separated by comma
%%   numbers-  selects numerical citations
%%   super  -  numerical citations as superscripts
%%   sort   -  sorts multiple citations according to order in ref. list
%%   sort&compress   -  like sort, but also compresses numerical citations
%%   compress - compresses without sorting
%%
%% \biboptions{comma,round}

% \biboptions{}

% if you have landscape tables
\usepackage[figuresright]{rotating}

% put your own definitions here:
%   \newcommand{\cZ}{\cal{Z}}
%   \newtheorem{def}{Definition}[section]
%   ...

% add words to TeX's hyphenation exception list
%\hyphenation{author another created financial paper re-commend-ed Post-Script}

% para poder introducir varias figuras que ocupen el ancho de las dos columnas.
\usepackage{subfigure}

% declarations for front matter


\begin{document}

\begin{frontmatter}

%% Title, authors and addresses

%% use the tnoteref command within \title for footnotes;
%% use the tnotetext command for the associated footnote;

%% use the fnref command within \author or \address for footnotes;
%% use the fntext command for the associated footnote;

%% use the corref command within \author for corresponding author footnotes;
%% use the cortext command for the associated footnote;
%% use the ead command for the email address,
%% and the form \ead[url] for the home page:
%%
%% \title{Title\tnoteref{label1}}
%% \tnotetext[label1]{}
%% \author{Name\corref{cor1}\fnref{label2}}
%% \ead{email address}
%% \ead[url]{home page}
%% \fntext[label2]{}
%% \cortext[cor1]{}
%% \address{Address\fnref{label3}}
%% \fntext[label3]{}

%\dochead{Cabecera artaculo}
%% Use \dochead if there is an article header, e.g. \dochead{Short communication}

\title{Influencia del ENSO en el nivel medio del mar del Pacífico Colombiano}

%% use optional labels to link authors explicitly to addresses:
%% \author[label1,label2]{<author name>}
%% \address[label1]{<address>}
%% \address[label2]{<address>}

\author[First]{Ayala, F.F. \corref{cor1}}
\ead{ffayalac@unal.edu.co}

\author[First,Second]{Delgado, J.K.}
\ead{jkdelgadog@unal.edu.co}

\author[First]{Osorio, A.F.}
\ead{afosorioar@unal.edu.co}

\cortext[cor1]{Autor en correspondencia.}

\address[First]{Grupo de investigación en oceanografía e ingeniería costera, OCEANICOS, Universidad Nacional de Colombia, Sede Medellín, Colombia}
\address[Second]{Coastal solutions fellows program, Cornell Lab of Ornithology, Cornell University, USA}


\begin{abstract}
%% Text of abstract

El nivel del mar es una de las variables oceanográficas más importantes en la gestión costera, debido a que condiciona las operaciones portuarias, los asentamientos humanos en la costa, la geomorfología de las playas y el hábitat de diferentes especies animales y es regulado por diversos fenómenos (oleaje, huracanes, tormentas, expansión térmica, glaciaciones) en una escala temporal amplia. 

A pesar de que el océano tiene su propia oscilación natural que condiciona el nivel del mar y la tasa de aumento de este es de 2.4 mm/año, según los registros de la última década, se ha demostrado que existen otros aumentos importantes, inducidos por fenómenos de gran escala como el ENSO, respaldados por correlaciones cercanas a 0.7 entre la serie global de nivel del mar y el índice climático MEI. 

En la zona de estudio, restringida al Pacífico Colombiano se han obtenido incrementos de hasta 30 cm de aumento/descenso del nivel del mar en las fases positiva/negativa del ENSO mediante el uso de técnicas de descomposición armónica y análisis espectral de las series de los mareógrafos, la transición gradual hasta estos ascensos máximos y su respectiva disipación suceden en períodos aproximados de 3 meses, lo que indica oscilaciones cada vez más energéticas e incluso comparables con el nivel del mar esperado en plazos de tiempo cercanos a lo 50 años. 

Esta influencia del ENSO en el nivel medio del mar, sugiere el uso de una variable oceanográfica usada en los índices climáticos que lo caracterizan (ONI, MEI, SOI) con el fin de monitorear eventos ENSO y su posible llegada a la costa Pacífica Colombiana

\end{abstract}

\begin{keyword}
%% keywords here, in the form: keyword \sep keyword

%% MSC codes here, in the form: \MSC code \sep code
%% or \MSC[2008] code \sep code (2000 is the default)

Nivel medio del mar \sep ascensos máximos \sep ENSO

\end{keyword}

\end{frontmatter}

%%
%% Start line numbering here if you want
%%
% \linenumbers

%% main text


\end{document}

%%
%% End of file `ejemplo latex RIAI.tex'.