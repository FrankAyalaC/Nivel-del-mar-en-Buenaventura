\chapter{Conclusiones y recomendaciones}
\section{Conclusiones}

El ENSO es el fenómeno más importante en la variabilidad interanual del nivel del mar en la bahía de Buenaventura y durante sus fases cálidas se presentan sobrelevaciones de hasta 35 cm sobre el nivel medio del mar y descensos de hasta 40 cm. El comportamiento del nivel del mar puede variar entre eventos niño y niña y es función de la duración dela fase cálida o fría, la intensidad del evento. El promedio de las sobrelevaciones ocurridas en los meses niño es de 43.3 cm y el promedio de los descensos durante los meses niña es de 45.2 cm. El promedio de las duraciones de los eventos ENSO es menor al promedio de las duraciones de los eventos Niña, esto puede verse explicado porque las fases frías del ENSO son más estables que las cálidas \cite{Gouirand2003}.

El ENSO puede aportar hasta 10 cm a la sobrelevación del nivel del mar, cómo aparece en ciertos registros en el evento Niño de 1982-1983, e incluso, es posible que ciertas condiciones amplifiquen o no estas sobrelevaciones, convirtiéndose en una amenaza para las comunidades costeras. Las sobrelevaciones y descensos del nivel del mar debidos solamente a fenómeno ENSO actúan en conjunto con la climatología anual del nivel y otras condiciones específicas que pueden intensificar y/o sostener dichos aumentos/descensos.

El patrón espacio-temporal más importante del nivel del mar en la escala interanual en el Pacífico Tropical es el ENSO y la varianza asociada a la banda espectral en la que se estudia puede representar hasta el 50\% de la varianza total del nivel del mar en regiones cercanas al trópico

Las aumentos persistentes e intensos del nivel del mar durante los eventos Niño logran atravesar el Pacífico de Oeste a Este y modificar las condiciones de nivel del mar en las costas de Suramérica y en determinadas ocasiones, generar posteriores descensos del nivel del mar.

Es posible la implementación de herramientas de pronóstico de los aportes del ENSO al nivel del mar en la bahía de Buenaventura con una red neuronal artificial construida y entrenada que predice el 30\% de los datos (conjunto de datos de validación) con un RMSE de 0.01 m y una correlación de Spearman igual a 0.83. Su estructura se resume en: 2 variables de entrada, nivel del mar y velocidad longitudinal de las corrientes, 2 nodos en una capa intermedia oculta y una capa de salida, las caracerísticas de su configuració son una función de activación tangente hiperbólica y el uso de corrección del sesgo.


\section{Recomendaciones}

Futuras investigaciones pueden involucrar el estudio y la correlación del nivel del mar con el principal forzador de la superficie libre del océano, el viento. Es posible que se determinen zonas con correlaciones rezagadas altas y suficientes para determinar una serie predictora determinante para el entrenamiento de la RNA, inclusive, puede buscarse información con registros temporales más extensos para calibrar mejor la red durante la etapa de entrenamiento y obtener mejores rendimientos durante la etapa de validación

Conocido el aporte del ENSO al nivel del mar en Buenaventura, quedan interrogantes acerca de las condiciones previas que deben existir tanto para que dichos aumentos y/o descensos se amplifiquen o sean más duraderos, como para que tengan efectos menores en la costa. 

El estudio anterior puede ampliarse a otras regiones del Pacifico colombiano como el municipio de Tumaco, Nariño, dónde también se han registrado aumentos súbitos del nivel mar. El modelo aplicado se debería mejorar para realizar una herramienta tecnológica que ayude a tomar decisiones rápidas para la gestión del riesgo en la costa.