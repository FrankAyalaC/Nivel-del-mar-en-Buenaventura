\chapter{Metodología}

\section{Datos empleados}

a) Datos de mareógrafo: En la bahía de Buenaventura  se han captado registros del nivel del mar mediante un mareógrafo ubicado en 3.8906 \textdegree N, -77.0808 \textdegree W. Este equipo radar consta de un transmisor que lanza un haz electromagnético hacía la superficie libre con una frecuencia modulada y bajo el efecto doppler, determina la distancia existente entre el transmisor y la superficie libre. Los errores que se pueden obtener bajo este método son debidos a cambios en la temperatura del aire, la interferencia de objetos en el haz e inclusive a la acción de las olas \cite{UNESCO2016}. Este mareografo hace parte de la red de estaciones del nivel del mar creada por la comisión oceanográfica intergubernamental de la UNESCO y sus registros del nivel del mar están disponibles desde 1953 hasta 2014, a resolución horaria. Debido a operaciones de mantenimiento y reparación, existen datos faltantes a largo del registro que han generado problemas en el tratamiento de la información explicados posteriormente. La información está disponible en la página web \url{www.ioc-sealevelmonitoring.org}.

b) Datos de reanálisis: Diferentes centros de análisis de información meteorologíca y oceánica como el ECFMW y el CMEMS, desarrollan y ejecutan diferentes modelos para describir la evolución espacio-temporal de variables termodinámicas como temperatura, salinidad, concentración de hielo, nivel del mar, etc. CMEMS usó el modelo de circulación global oceánica NEMO en conjunto con técnicas de reanálisis para generar un producto llamado GLORYS12V1, allí se puede obtener información del nivel del mar a partir de datos altimétricos desde 1993 hasta 2018 \cite{Fernandez2018}. La resolución espacial es de 1/12\textdegree x 1/12\textdegree y la resolución temporal es mensual. La región de descarga está comprendida entre -10\textdegree N y 10\textdegree N y entre 130\textdegree E y -74\textdegree
W y se eligió así para evitar problemas de borde cuando se construyan las funciones empíricas ortogonales. La información está disponible en la página web \url{marine.copernicus.eu}.

\section{Análsis exploratorio de los datos}

\subsection{Análisis de la información del mareógrafo}

Los registros del mareógrafo de Buenaventura se utilizan para conocer la evolución temporal del nivel del mar en la costa, y en este caso, sus aumentos/descensos durante las fases cálidas/frías del ENSO. Por lo tanto, se debe iniciar por identificar qué componente de la marea es útil en este estudio.

La marea puede dividirse en dos componentes: marea astronómica y marea meteorológica. La marea astronómica se deriva de la atracción gravitatoria en el sistema Tierra-Sol-Luna y por lo tanto, es fácil precisar su aporte en cualquier región del mundo. Por otro lado, la marea meteorológica o residual se debe tanto a fenómenos microclimáticos como macroclimáticos, que a su vez ocurren en escalas temporales diferentes, dificultando su predicción.

La marea astronómica de una serie se obtiene con una técnica llamada descomposición armónica \cite{Dronkers1975}, esta técnica descompone la señal de marea en diferentes señales (componentes armónicas) que tienen amplitud, período y fase asociados. Estas componentes armónicas en conjunto, explican la marea astronómica total, la cuál se debe a la acción de la fuerza generadora de marea, originada por el desbalance entre la fuerza de atracción gravitacional entre la tierra y otros cuerpos celestes y la fuerza centrífuga.

\textcolor{red}{tipos de marea del pauloski}

Existen herramientas computacionales que realizan la descomposición armónica de la marea, algunas de estas están estructuradas en MATLAB o Python \cite{Pawlowicz2002} y para su implementación correcta no pueden existir muchos datos faltantes, debido a que la identificación de la fase y amplitud de los armónicos se vuelve imprecisa. 

Inicialmente, se desiste de usar todo el registro del mareógrafo por tener años con hasta 80\% de datos faltantes que no permiten que la herramienta T\_TIDE estime bien la amplitud de los armónicos más importantes (ej: el armónico $M_{2}$ debería tener una amplitud de 150 cm, pero sólo reconoce 105 cm); como alternativa, se usan los 18.6 años más completos dentro de la serie, debido a que cada período de estos, la declinación lunar cumple un ciclo llamado ciclo nodal lunar y las fuerzas generadoras de marea pasan por una fase completa, aun con esta acotación de la serie la descomposición armónica siguió siendo imprecisa. Mientras se decidía usar los 6 años de registro continuo con los que se cuenta y obtener el comportamiento de los armónicos para luego extrapolarlos para todo el registro (asumiendo la pérdida de información de las componentes que sólo ocurren en el largo plazo) se conoció una herramienta más precisa y y más reciente para realizar la descomposición armonica, PyTides. \textcolor{red}{CITAR}

Con la marea astronómica se obtiene la serie de marea residual, restándola de la serie medida por el mareógrafo. Después se centra en su valor medio para posteriormente, hacer referencia a los ascensos como \textbf{sobrelevaciones del nivel medio del mar}. Con esta serie se realizarán los análisis posteriores.

Adicionalmente, se calcula la serie mensual del mareógrafo realizando promedios mes a mes con el objetivo de compararla con la serie representativa de la región costera elegida en el siguiente literal.

\subsection{Análisis de la información de CMEMS} \label{sec:1}
Los registros de nivel del mar en cada píxel se centran en su media, al igual que se hizo con la marea residual, con el objetivo de comparar las magnitudes de las sobrelevaciones con las registradas por el mareógrafo.

Posteriormente, se determina una región costera frente a la bahía de Buenaventura y por ende, de la isla Punta Soldado, en este caso el criterio para elegir su ancho longitudinal es marcado por la isobara de 200 m \cite{Montagut2012}. Su ubicación está entre -3.65\textdegree N y 3.85\textdegree N y entre -77.6\textdegree W y -77.2\textdegree W, después se promedia espacialmente el nivel medio del mar para obtener una serie representativa, usada posteriormente en correlaciones con otras regiones del océano. Esta serie también se compara con la serie mensual de los datos medidos por el mareógrafo para validar su comportamiento.

\textcolor{red}{Con el objetivo de visualizar espacialmente las sobrelevaciones del nivel del mar durante un año catalogado como Niño o Niña según el índice ONI, se grafica en un mapa de contorno, la altura del nivel del mar, Al tener mucha variación, se decide interpolar a una resolución con el fin de mejorar la visualización}

\section{Caracterización del nivel del mar durante eventos ENSO}

\textcolor{red}{Al haber especial interés en los valores máximos }La resolución de la serie residual o meteorológica del mareógrafo es muy alta para el análisis del comportamiento del nivel durante los eventos ENSO, por lo tanto, se decide calcular su envolvente diaria, es decir, las \textbf{sobrelevaciones máximas diarias}. Esta se grafica en conjunto con los períodos de tiempo dónde se presentaron fases cálidas y frías del ENSO seǵun el índice ONI y el índice MEI, con el objetivo de conocer su comportamiento durante esos meses. Se suavizó la serie de sobrelevaciones en ventanas de 90 días para captar mejor las variaciones del nivel del mar.

Con la información del CMEMS, se grafica un diagrama de Hovmöller para conocer la evolución temporal de las sobrelevaciones del nivel del mar durante las fases cálidas y frías del ENSO y compararla con la información (\textbf{CUAL INFO?}) del mareógrafo. Finalmente, con la relación de la información espacial y temporal, se determina la duración y frecuencia de los eventos ENSO registrados, las épocas del año en las que se presentaron y las sobrelevaciones que ocurrieron. 

\section{Correlacion del nivel del mar con índices macroclimáticos}

\section{Efecto del ENSO en la variabilidad del nivel del mar}
	
Para estudiar el aporte del ENSO en la variabilidad del nivel del mar, se analiza la escala interanual a través de dos procedimientos diferentes. Por un lado, para cada píxel, se filtra la serie de nivel medio del mar en la banda de interés del espectro de potencias de Fourier, definida entre 2 años y 6 años, con el objetivo de obtener sólo la variabilidad en la escala temporal dónde el ENSO ocurre. Posteriormente se determinan los patrones espacio-temporales  más dominantes a través de una función ortogonal empírica (EOF por sus siglas en inglés) y se seleccionan la componente principal que más aporta a la varianza junto con su modo de oscilación temporal. Finalmente, se compara esa componente principal con un índice macroclimático del ENSO, en este caso, el índice oceánico del niño (ONI por sus siglas en en inglés). 

Por otro lado, se calcula para cada píxel, la varianza total y la varianza asociada a la banda de interés especificada anteriormente y con ello, se determinan los mapas de varianza, varianza en la banda y porcentaje de varianza. Estos mapas permiten identificar las regiones dónde la varianza en la escala interanual aporta más o menos a la varianza total.

\section{Predicción del nivel del mar en una zona costera de interés}

A fin de encontrar una región que permita pronosticar el comportamiento del nivel del mar en la zona costera elegida en el literal \ref{sec:1}, se realizan mapas para toda la región dónde en cada píxel se calcula la correlación de Pearson y Spearman entre la serie representativa de la zona costera rezagada 1, 2 y 3 meses y las series, tanto de nivel del mar, como de temperatura, velocidad de las corrientes en la longitud y velocidad de las corrientes en la latitud. Para cada variable, se obtiene diferentes zonas dónde se presentan las mayores correlaciones y en ellas se realiza un promedio espacial para obtener las series que funcionen como predictoras del nivel del mar en la zona costera.

Finalmente, se construye y entrena una red neuronal con la función tangente hiperbólica como función de activación y con \textbf{correción del sesgo}. Para optimizar estas predicciones de nivel del mar en la zona costera, a partir de las series de nivel del mar en una región mar afuera \textcolor{red}{se remueve la información asociada al ciclo anual, al estandarizar las series } y escalarlas para que estén entre el rango de -1 a 1.

\textcolor{blue}{Cómo ligarlo al ENSO y entender bien lo de los rezagos.}
