\chapter{Introducci\'{o}n}

\textbf{Problema y justificacion}

A causa del aumento continuo del nivel del mar, algunas comunidades del Pac\'{i}fico Colombiano han sufrido la p\'{e}rdida de los territorios costeros donde est\'{a}n asentados y de sus recursos marinos. es importante conocer factores que lo generan para mejorar la gestion del riesgo en la costa. (\textbf{explicar la impoirtancia y de su uso por la comunidad})

Aunque la tendencia global del nivel del mar tiene valores conocidos que oscilan entre los 2-3 mm/ano, no es un indicador espec\'{i}fico de los cambios del nivel que ocurren local o regionalmente. Por ejemplo, en registros del mare\'{o}grafo de Buenaventura, a tendencia anual es de 2.8 mm/ano, pero se notan aumentos s\'{u}bitos del nivel de hasta 30 cm durante algunos meses.

Estos cambios locales de nivel del mar se deben a procesos est\'{a}ticos y din\'{a}micos: Patrones de viento locales, fen\'{o}menos clim\'{a}ticos locales, expansi\'{o}n t\'{e}rmica,derretimiento de masas de hielo, reservorios de agua en tierra y otros fen\'{o}menos macroclim\'{a}ticos. (partir en 2), Estos fen\'{o}menos se presentan en diversas escalas de tiempo, con intensidades y efectos diferentes sobre la superficie libre.

A pesar de todo este conjunto de factores, aumentos s\'{u}bitos importantes del nivel, se han  asociado a fen\'{o}menos de variabilidad interanual, como lo puede ser el ENSO, la PDO, la NAO, entre otros.  \textbf{cita de un estudio global con otros \'{i}ndices macroclim\'{a}ticos}

Los efectos del ENSO se han empezado a documentar desde... efectos del ENSO

Hay fenómenos macroclimáticos y ocales que cambian el nivel del mar tales como: deshielo glaciar, expansión térmica, oscilaciones naturales o régimen local de vientos. (\textbf{CITAR}) 




También se determina la 


\textbf{Casos hechos pero con leves faltas}

Es el caso de Estos manes quienes registraron aumentos del nivel que llevaron a la reubicaci\'{o}n de diversas comunidades en el pac\'{i}fico colombiano debidos a fases c\'{a}lidas del ENSO (El Nino) que movilizaron masas de agua y trasladaron la línea de costa, tierra adentro. \textbf{explicar lo que genero}. Aunque an\'{a}lisis espaciales de las anomal\'{i}as de nivel del mar y su relaci\'{o}n con este fen\'{o}meno macroclim\'{a}tico no han sido reportados.

Por otro lado, diversos autores ha estudiado la variabilidad espacio-temporal de las anomal\'{i}as de nivel del mar y sus principales forzadores en el Pac\'{i}fico Tropical y además han estimado estas anomal\'{i}as en la ocurrencia de diferentes eventos ENSO fuertes. Se han encontrado correlaciones positivas fuertes en el tr\'{o}pico, pero que se van debilitando a medida que se var\'{i}a latitudinalmente. El estudio de una region específica en el Pacífico Central...justificar

- Hablar de los hotspots

- Hablar de los objetivos

- Hablar de la ruta metodol\'{o}gica  (párrafos por capítulo)

En la introducci\'{o}n, el autor presenta y se\~{n}ala la importancia, el origen (los antecedentes te\'{o}ricos y pr\'{a}cticos), los objetivos, los alcances, las limitaciones, la metodolog\'{\i}a empleada, el significado que el estudio tiene en el avance del campo respectivo y su aplicaci\'{o}n en el \'{a}rea investigada. No debe confundirse con el resumen y se recomienda que la introducci\'{o}n tenga una extensi\'{o}n de m\'{\i}nimo 2 p\'{a}ginas y m\'{a}ximo de 4 p\'{a}ginas.\\


La redacci\'{o}n debe ser impersonal y gen\'{e}rica. La numeraci\'{o}n de las hojas sugiere que las p\'{a}ginas preliminares se realicen en n\'{u}meros romanos en may\'{u}scula y las dem\'{a}s en n\'{u}meros ar\'{a}bigos, en forma consecutiva a partir de la introducci\'{o}n que comenzar\'{a} con el n\'{u}mero 1. La cubierta y la portada no se numeran pero si se cuentan como p\'{a}ginas.\\

Para resaltar, puede usarse letra cursiva o negrilla. Los t\'{e}rminos de otras lenguas que aparezcan dentro del texto se escriben en cursiva.\\