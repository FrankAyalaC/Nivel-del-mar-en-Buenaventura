\chapter{Introducción}

En el Pacífico Colombiano se han registrado aumentos súbitos del nivel del mar que han provocado la pérdida de los territorios costeros dónde diversas comunidades estaban asentadas, estos ascensos locales pueden deberse a la acción conjunta de procesos geológicos, depositación de sedimentos, patrones locales de vientos y fenómenos macroclimáticos \citep{Church2019}. Algunos autores asociaron sobrelevaciones del nivel de hasta 30 cm con la ocurrencia de El Niño-Oscilación del Sur (ENSO) \citep{Correa2000}. Sin embargo, aún no se caracterizan los aportes de este fenómeno dentro de la variabilidad del nivel del mar. 

\emergencystretch 3em

La tasa de aumento del nivel del mar debida a cambios eustáticos o globales, ha oscilado entre 2 mm/año y 3 mm/año desde el siglo pasado \citep{Cazenave2014}, estos cambios son ocasionados por fenómenos como expansión térmica \citep{Lombard2005}, oscilaciones climáticas, derretimiento de masas de hielo y aporte de aguas contenidas en reservorios terrestres \citep{Cazenave2008}. Adicionalmente, estudios previos demostraron correlaciones entre el nivel medio del mar global en la escala interanual y el almacenamiento global de agua terrestre durante el ENSO para la región norte del Pacífico tropical \citep{Cazenave2012}. Los niveles del mar en las latitudes bajas tienen una mayor respuesta al ENSO que las latitudes medias y altas \citep{Soumya2015}, a pesar de esos impactos en las regiones de baja latitud, no ha aumentado la tendencia del nivel \citep{Chang2013}

Han existido grandes esfuerzos por entender los efectos del ENSO en la variabilidad espacio-temporal del nivel del mar en el Pacífico Tropical \citep{Zhang2012} y su respectiva relación el viento, presión a nivel del mar y corrientes superficiales \citep{Delcroix1994}, \citep{Gill1983}. Adicionalmente, han resaltado las sobrelevaciones del nivel que genera una fase cálida del ENSO en el Pacífico Este. Por ejemplo, en el Este de California y en el Pacífico Noroeste, se han presentado aumentos de hasta 11 cm debidos a fases cálidas del ENSO junto a porcentajes mayores de erosión costera \citep{Cayan2008}, \citep{Barnard2015}, también se han registrados aumentos súbitos del nivel del mar en la bahía de Buenaventura del Pacífico colombiano, las cuales han alertado las comunidades de la zona y han generado la necesidad de investigar su duración, amplitud y frecuencia así como las posibilidades de predecirlas antes de que arriben a la costa \citep{Valle-Levinson2017} 

\newpage

El objetivo de este estudio es conocer el efecto del ENSO en la variabilidad interanual del nivel del mar en la zona costera de Buenaventura su relación con aguas del océano Pacífico tropical, caracterizar su comportamiento en eventos Niño y Niña \citep{Wang2018} y adicionalmente, identificar regiones y variables en zonas alejadas de la costa que permitan la predicción de los aumentos y descensos que sólo se deben al fenómeno ENSO.

Para este estudio, se ha encontrado que en la Bahía de Buenaventura el promedio de las sobrelevaciones ocurridas durante los meses niño es de 43.3 cm y que durante los meses niña el promedio de los descensos es 45.2 cm. Los aportes sólamente por fenómeno ENSO pueden ser de hasta 10 cm (Niño 1982-1983), los cuales pueden predecirse a partir del nivel del mar y las corrientes zonales en otras regiones del Pacífico a partir del uso de una red neuronal artifical (RNA) con un RMSE de 0.01 m y un coeficiente de correlación de Spearman de 0.83. 

En el capítulo 1 se desarrolló el fundamento teórico de las técnicas de análisis de datos y herramientas de predicción empleadas en el presente estudio: la descomposición armónica, el análisis espectral, las funciones ortogonales empíricas y las redes neuronales artificiales (RNA's)

En el capítulo 2 se expone la metodología que se usará para la aplicación coherente de las diferentes técnicas nombradas en el capítulo 1 para la caracterización de la variabilidad del nivel del mar desde la bahía de Buenaventura hasta el Pacífico Tropical, adicionalmente se describen la fuentes de información empleados en esta investigación.

El capítulo 3 presenta los resultados después de la aplicacion de las técnicas y herramientas propuestas en el capítulo 2, desde la marea residual hasta la implementación de la red neuronal artificial construida para la predicción del nivel del mar debido al ENSO en la zona costera de la bahía de Buenaventura. Se resalta la comparación de las fuentes de información del mareógrafo y la información espacial con el índice ONI, así como las zonas elegidas para la determinación de las series temporales de las variables predictoras del nivel.

En el capítulo 4 se presentan las conclusiones del presente estudio, se resumen y comentan los resultados obtenidos, sus posibles causas y consecuencias y sus limitaciones. Adicionalmente, se presentan las posibilidades y recomendaciones para los trabajos futuros.

