\chapter{Objetivos}
Conocer el efecto del ENSO en la variabilidad interanual del nivel del mar en la zona costera de Buenaventura.

\section{Objetivos específicos}

\begin{itemize}
	\item Identificar la relación del nivel del mar en Buenaventura con las fases cálidas y frías del ENSO a través del índice ONI.
	\item Caracterizar los eventos Niño y Niña en términos de su duración y amplitud así como desde el comportamiento del nivel del mar en Buenaventura.
	\item Analizar espectralmente la serie mensual de nivel del mar para la información del mareógrafo y filtrarla en la banda de interés (2-6 años)
	\item Conocer la variabilidad espacial del nivel del mar en el Pacífico Tropical a través de mapas de varianza y funciones ortogonales empíricas.
	\item Caracterizar espacialmente el nivel del mar durante la ocurrencia de eventos Niño y Niña.
	\item Determinar una región costera frente a la bahía de Buenaventura a partir de la información espacial y su respectiva serie de nivel del mar promediada espacialmente
	\item Elegir zonas de mayor correlación rezagada entre el nivel del mar, temperatura, corrientes zonales, corrientes meridionales y el nivel del mar en la costa.
	\item Predecir la variación interanual del nivel del mar en la costa a partir del uso de una red neuronal artificial
\end{itemize}