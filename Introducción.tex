\documentclass[11pt]{beamer}

%%%%%%%% tema e cor %%%%%%%%
\mode<presentation> {
\usetheme{Madrid}
\usecolortheme{beaver}
}
\usepackage[spanish]{babel}
\usepackage[utf8]{inputenc}
\usepackage{graphicx} 
\usepackage{booktabs} 
\usepackage[document]{ragged2e}



\institute[unalmed] 
{
	%================= logos no meio =====================
	\vspace*{-0.35cm}
	\includegraphics[scale=0.3]{img/logo-unal.png}
	\vspace*{0.35cm}\\
	Universidad Nacional de Colombia. Sede Medellín
	%\medskip
	%\texttt{\{lods.eng,ronety\}@uea.edu.br} % emails
}
\date{\today}




\begin{document}
%%%%%%%% título y subtítulo %%%%%%%%
\title[Presentación parcial]{Quiz 1. Estabilidad de laderas} 

%%%%%%%% nombre de autores %%%%%%%%
\author[Franklin Farid Ayala Cruz]{Franklin Farid Ayala Cruz} 

\begin{frame}
	\titlepage 
\end{frame}

\begin{frame}{Introducción}

\justify
La múltiple cantidad de derrumbes existentes en el país y la falta de apropiación del conocimiento técnico acerca de la estabilidad de laderas ha generado mucha incertidumbre respecto a los factores que ocasionan dichos fenómenos. Por lo anterior, se han generado falsos conceptos, casi creencias, que no explican de forma fidedigna el origen de dichas fallas, un ejemplo de estos podrán ser:

\begin{enumerate}
	\item La principal razón por la que falla un talud es por la lluvia
	\item Según la cobertura vegetal del talud hay más probabilidad que falle o no
\end{enumerate}

\end{frame}

\begin{frame}
	\frametitle{Contenido} 
	\tableofcontents 
\end{frame}

%%%%%%%% slides %%%%%%%%
\section{Delizamiento de Villatina} 
\subsection{Causas}
\subsection{Conclusiones}

\begin{frame}{Deslizamiento en Villatina}

\begin{columns}[t]
	\column{0.5\textwidth}
	
	Villatina es un barrio ubicado en la ladera centro-oriental de Medellín, en la comuna 9 (Villa hermosa) y que en 1987 sufrió un deslizamiento con magnitudes y consecuencias grandes que no han tenido un par en la ciudad. Dentro de algunas cifras están:
	
	
	\begin{itemize}
	\item Cantidad de tierra removida 20.000 $m^{3}$
	
	\item Aproximadamente 562 personas muertas, 1700 personas damnificadas y 100 viviendas destruidas.
	
	\end{itemize}
	\column{0.5\textwidth}
	Sus causas, aún fruto de estudio, se enmarcan dentro de 4 grandes factores: geológicos, topográficos, antrópicos y meteorológicos. Aunque hay una hipótesis acerca de la concentración de flujos cercanos al Cerro Pan de Azúcar y otra acerca de la existencia de posibles "caletas" de explosivos en la zona.

	\begin{figure}
		\includegraphics[width=\textwidth]{img/v1.jpg}
	\end{figure}

\end{columns}
\end{frame}

\begin{frame}{Causas}
	\begin{columns}[t]
		
		\column{0.5\textwidth}
		
		\justify
		La radiografía de este desastre demuestra la existencia de una acequía en la zona superior de la ladera y que debido a la infiltración de agua que esta permitía generó el desprendimiento de una gran masa de tierra. Esta explicación puede obtenerse claramente en la imagen y fue sustentada por el profesor en clase.
		
		\column{0.5\textwidth}
		
		\begin{figure}
			\includegraphics[width=\textwidth]{img/v2.jpg}
		\end{figure}
\end{columns}	

\end{frame}

\begin{frame}{Conclusiones}
	\begin{columns}[t]
	
	\column{0.5\textwidth}
	
	\justify
	La información en los diferentes SIG, el régimen de lluvias intenso de la época y las imágenes obtenidas, permiten inferir que sí hubo una mayor descarga infiltrada por parte de la acequía de dicha zona y por lo tanto, la remoción de esa cantidad de tierra. 
	
	Por lo anterior, se está de acuerdo con el profesor acerca de la causa de este deslizamiento. 
	
	\column{0.5\textwidth}
	
	\begin{figure}
		\includegraphics[width=\textwidth]{img/v3.jpg}
	\end{figure}
\end{columns}	

\end{frame}

\section{Deslizamiento de Alto Verde} 
\subsection{Causas}
\subsection{Conclusiones}

\begin{frame}{Deslizamiento de Alto Verde}
	\begin{columns}[t]
	\column{0.5\textwidth}
	
	El barrio Alto verde está ubicado en la Comuna N°14 (El Poblado) de la ciudad de Medellín. Allí, el domingo 17 de Noviembre de 2008, aproximadamente 45.000 $m^{3}$ de tierra se desprendieron en un movimiento de masa rotacional dejando el siguiente saldo:
	
	\begin{itemize}
		\item 12 personas muertas
		\item 6 viviendas destruídas
	\end{itemize}
	
	\column{0.5\textwidth}
	
	\begin{figure}
		\includegraphics[width=\textwidth]{img/av1.jpeg}
	\end{figure}
	
	\end{columns}
\end{frame}

\begin{frame}{Causas}
\begin{columns}[t]

	\column{0.5\textwidth}
	\justify
	La principal causa expuesta en la clase fue la presencia de un canal que conducía agua hacía un tanque en la parte alta del deslizamiento y que por su mal estado de operación y mantenimiento dejaba filtrar aguas hacia el talud. Una evidencia de esto es la forma de parábola invertida que el deslizamiento 

	\column{0.5\textwidth}

\begin{figure}
	\includegraphics[width=\textwidth]{img/av2.png}
\end{figure}
\end{columns}	

\end{frame}

\begin{frame}{Conclusiones}
\begin{columns}[t]
	
	\column{0.5\textwidth}
	
	\justify
	Según los estudios realizados por el SIMPAD y la firma consultora INTEINSA, justamente la mala conducción de aguas hacia el tanque en cuestión y la altura del talud, generaron el movimiento en masa. También se aduce la presencia de suelos residuales derivados de dunita, suelos susceptibles a estos eventos cuando están saturados.
	
	Por lo anterior, se está de acuerdo con el profesor acerca de la causa de este deslizamiento. 
	
	\column{0.5\textwidth}
	
	\begin{figure}
		\includegraphics[width=\textwidth]{img/av3.png}
	\end{figure}
\end{columns}	

\end{frame}

\section{La Huesera} 
\subsection{Sector 1 y Sector 3}
\subsection{Sector 2}
\subsection{Sector 4 y Sector 5}

\begin{frame}{La Huesera}
	En el suroeste antioqueño, existe un sector donde se quemaban huesos llamado "La huesera". Esta zona ha generado diferentes problemas de movilidad en esta región del departamento debido a la ocurrencia de diversos deslizamientos
	
	\begin{figure}
		\includegraphics[scale=0.17]{img/lh1.jpg}
	\end{figure}
\end{frame}

\begin{frame}{Sector 1 y Sector 3}
	En estos dos sectores se puede evidenciar la concentración de flujo en la parte superior de los taludes, una muestra de esto es la forma parabólica de los deslizamientos. El flujo de acumulación se pudo haber generado por una quebrada cercana o un afloramiento.\\
	\vspace{5mm}
	\textbf{\large{Conclusión}}\\
	 
	Se aceptan las causas expuestas en clase acerca de estos deslizamientos, puesto que es irrefutable el mecanismo de falla que presentaron
\end{frame}

\begin{frame}{Sector 1 y 3}
	\begin{columns}[t]
	\column{0.5\textwidth}
	Sector 1
	\begin{figure}
		\includegraphics[width=\textwidth]{img/lh2.jpg}
	\end{figure}
	
	\column{0.5\textwidth}
	Sector 3
	\begin{figure}
		\includegraphics[width=\textwidth]{img/lh3.jpg}
	\end{figure}
	\end{columns}
\end{frame}

\begin{frame}{Sector 2}

En el sector 2 existe una particularidad a señalar y es la ubicación de una zona de extracción de material rocoso y demás (canteras). Estos sitios debilitan la cobertura vegetal de los taludes y facilitan la infiltración de aguas. Es importante recalcar no ha habido acción de las autoridades acerca de la operación de dicha cantera\\
\vspace{5mm}
\textbf{\large{Conclusión}}\\ 

La acción de las máquinas que explotan el material del talud pueden alterar el comportamiento mecánico del talud de una forma sustancial, sobre todo respecto a la cantidad de agua que puede infiltrarse en cada uno de ellos. En esta vía a la Huesera no debe permitirse el trabajo de dichas compañías, dada la susceptibilidad de la zona en términos geológicos e hidrológicos.

\end{frame}

\begin{frame}{Sector 2}
	\begin{columns}[t]
	\column{0.5\textwidth}
	\begin{figure}
		\includegraphics[width=\textwidth]{img/lh4.jpg}
	\end{figure}
	
	\column{0.5\textwidth}
	\begin{figure}
		\includegraphics[width=\textwidth]{img/lh5.jpg}
	\end{figure}
\end{columns}
\end{frame}
\end{frame}

\begin{frame}{Sector 4 y Sector 5}

En los sectores 4 y 5 se encuentran 2 condiciones muy importantes y que son claves para la identificación de las problemáticas en estas zonas. La primera condición es la existencia de caminos riales en las cabeceras de dichos taludes, los cuales condicionan y determinan el tránsito de aguas superficiales y subterráneas. La segunda condición es la acumulación de dichos flujos los cual favorece los \\

\vspace{5mm}
\textbf{\large{Conclusión}}\\ 

Se está de acuerdo con las hipótesis plantedas por el profesor Frank Montoya respecto al flujo de acumulación y la conducción de dicho flujo por parte de los caminos riales, sin embargo, es necesaria más información al respecto acerca de la ubicación del talud derrumbado y de sus posibles detonantes, puesto que las imágenes existentes no suministran las suficientes herramientas para aseverar definitivamente dicha hipótesis.
\end{frame}


\begin{frame}{Sector 4 y 5}

	\begin{columns}[t]
	\column{0.5\textwidth}
	Sector 4
	\begin{figure}
		\includegraphics[width=\textwidth]{img/lh6.jpg}
	\end{figure}
	
	\column{0.5\textwidth}
	Sector 5
	\begin{figure}
		\includegraphics[width=\textwidth]{img/lh7.jpg}
	\end{figure}
\end{columns}
\end{frame}




%%%%%%%% agradecimentos %%%%%%%%
%------------------------------------------------


%------------------------------------------------
%------------------------------------------------
%%%%%%%% referencias %%%%%%%%
%\nocite{*}
%\begin{frame}{Referencias}
%\bibliographystyle{unsrt}
%\bibliography{ref.bib}
%\end{frame}


%%%%%%%% repete primeiro slide %%%%%%%%
\begin{frame}
\titlepage 
\end{frame}


\end{document}